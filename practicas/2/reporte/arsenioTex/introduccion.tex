\section{Introducción}
El \textbf{desbordamiento de búfer} (\emph{buffer overflow}) es una de las vulnerabilidades de seguridad
más conocidas en el ámbito de la programación en C. Consiste en escribir datos más allá de los
límites previstos de un arreglo o búfer, lo que puede ocasionar fallos de segmentación o bien
sobrescritura de secciones de memoria críticas. Esta vulnerabilidad, cuando es explotada por
atacantes, les permite tomar el control del flujo de ejecución de un programa.

En esta práctica se estudiarán los fundamentos del desbordamiento de búfer y se procederá a
explotarlo en dos programas escritos en C que contienen funciones deliberadamente vulnerables.
Asimismo, se demostrará cómo es posible ejecutar código privilegiado (o acceder a información
sensible) sin cumplir los requisitos previstos en el programa. Para ello, se utilizarán herramientas
de depuración (\texttt{gdb}) y lenguajes de scripting (\texttt{python}) que facilitan la interacción
con la memoria y la construcción de los \emph{exploits}.

Finalmente, se plantearán métodos de mitigación y corrección de las vulnerabilidades, como el
uso de funciones seguras para lectura de cadenas (\texttt{fgets} en lugar de \texttt{gets}), la
inclusión de protecciones en la pila, y buenas prácticas de programación que ayuden a prevenir
este tipo de fallos.

\vspace{1em}
