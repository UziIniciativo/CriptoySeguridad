\section{Preguntas y Respuestas}

\begin{enumerate}

    \item \textbf{¿Qué significa que una prueba de penetración sea de caja blanca?}

    Una prueba de penetración de \textit{caja blanca} implica que el evaluador tiene acceso total
    al conocimiento del sistema o aplicación a probar, incluyendo código fuente, configuraciones,
    arquitectura de software, credenciales, etc. En esta práctica, contamos con el código fuente de
    los programas vulnerables, lo que nos permite comprender en detalle su funcionamiento interno
    y diseñar exploits de forma más directa. (véase \citep{cowan2002}).

    \item \textbf{¿Por qué es necesario aprender a explotar código?}

    Porque comprender cómo se explota el código inseguro permite:
    \begin{itemize}
        \item Identificar y mitigar de manera efectiva las vulnerabilidades.
        \item Validar la seguridad de un sistema, asumiendo el papel de un “atacante ético”.
        \item Crear conciencia sobre la importancia de escribir y mantener código seguro.
    \end{itemize}

    (Como discuten \citet{erickson2008} y \citet{howard2003}).

    \item \textbf{Menciona el significado de los siguientes registros: EAX, EBX, ECX, EDX, ESI, EDI, EBP, ESP, EIP.}

    Estos registros corresponden a la arquitectura x86 (y en versiones extendidas a x86-64, con nombres similares)(\citealp{foster2005}; \citealp{gupta2012}). A grandes rasgos:
    \begin{itemize}
        \item \textbf{EAX (Extended Accumulator Register)}: utilizado generalmente para operaciones aritméticas y como registro de retorno de funciones.
        \item \textbf{EBX (Extended Base Register)}: se emplea a menudo como base para cálculos de direcciones en memoria.
        \item \textbf{ECX (Extended Counter Register)}: se usa como contador en bucles e instrucciones repetitivas.
        \item \textbf{EDX (Extended Data Register)}: apoya a EAX en operaciones aritméticas de mayor tamaño (multiplicaciones y divisiones).
        \item \textbf{ESI (Extended Source Index)}: comúnmente se usa como puntero fuente en operaciones de cadena.
        \item \textbf{EDI (Extended Destination Index)}: análogo a ESI, pero para el destino.
        \item \textbf{EBP (Extended Base Pointer)}: señala el inicio del marco de pila de una función, facilitando acceso a parámetros locales.
        \item \textbf{ESP (Extended Stack Pointer)}: apunta al tope actual de la pila.
        \item \textbf{EIP (Extended Instruction Pointer)}: contiene la dirección de la instrucción que se ejecutará a continuación.
    \end{itemize}

    \item \textbf{¿Qué es \textit{little endian} y \textit{big endian}? ¿Cuál usa mi procesador? ¿Qué arquitectura tiene mi computadora?}

    El \textit{endianess} describe el orden en que los bytes se almacenan en memoria. En 
    \textit{little endian}, el byte menos significativo se almacena en la dirección de memoria 
    más baja; en \textit{big endian} sucede lo contrario.  
    \\
    Mi procesador (x86\_64) emplea \textit{little endian}, que es el formato común en la mayoría 
    de procesadores Intel y AMD de 64 bits en la actualidad.

    \item \textbf{¿Qué es un segmento y qué es un \textit{offset}?}

    En el contexto de las arquitecturas x86 clásicas, un \textit{segmento} es una región 
    lógica de memoria definida por un registro de segmento. El \textit{offset} es el 
    desplazamiento relativo dentro de ese segmento. Con la evolución a x86\_64, la segmentación 
    se redujo, pero estos conceptos se mantienen en la base de la arquitectura \citep{erickson2008}.

    \item \textbf{¿Qué realiza la instrucción \texttt{lea} en ensamblador?}

    La instrucción \texttt{lea} (\textit{load effective address}) carga en un registro la
    dirección efectiva de una operación de memoria en lugar de cargar el valor que reside en 
    esa dirección. Se usa para cálculos de direcciones y puede reemplazar operaciones de suma 
    o aritmética directa sin tocar la memoria \citep{erickson2008, howard2003}.

    \item \textbf{¿Crees que esta vulnerabilidad desapareció con las nuevas funciones seguras como \texttt{fgets}?}

    No. Usar funciones más seguras como \texttt{fgets} reduce la probabilidad de un desborda-
    miento, pero no lo elimina por completo. Las vulnerabilidades de desbordamiento siguen 
    presentes si no hay validaciones de longitud adecuadas o existen otros errores de manejo 
    de memoria. Por tanto, las \textit{buenas prácticas de programación} siguen siendo 
    esenciales para la seguridad.

    \item \textbf{Investiga tres casos famosos en los cuales se explotó la vulnerabilidad de desbordamiento de búffer. ¿Cuáles fueron sus repercusiones?}

    \begin{itemize}
        \item \textit{Morris Worm (1988)}: primer gusano ampliamente difundido, se basó en
              varios exploits (incluyendo el desbordamiento de \texttt{finger}), colapsando
              grandes partes de Internet de aquel entonces \citep{gallagher2014}.
        \item \textit{Blaster Worm (2003)}: atacaba el servicio RPC de Microsoft Windows,
              propagándose masivamente y ocasionando importantes pérdidas económicas \citep{cert2003}.
        \item \textit{Heartbleed (2014)}: vulnerabilidad en la implementación de OpenSSL,
              que si bien no era un \textit{overflow} clásico, explotó la mala validación de
              un búfer para leer datos sensibles de la memoria del servidor \citep{morrisworm2018}.
    \end{itemize}

    \item \textbf{¿Qué es el bug \textit{Off-by-One}? ¿Cómo funciona? ¿Cómo se previene?}

    Es un error en el manejo de índices o límites de un arreglo, en que se accede o sobrescribe 
    un elemento adicional (o se queda corto por uno). Puede provocar desbordamientos 
    inadvertidos y comportamientos inesperados. Para prevenirlo, hay que hacer validaciones 
    apropiadas de los límites y utilizar funciones o métodos que obliguen a especificar la 
    longitud real de los buffers\citep{howard2003}.

    \item \textbf{¿Se podría explotar la vulnerabilidad de desbordamiento de búfer usando Bash? Argumenta tu respuesta.}

    Es posible pasar \textit{inputs} maliciosos desde Bash a programas vulnerables en C 
    (encadenando el ataque). Sin embargo, explotar directamente la memoria del intérprete 
    Bash requeriría un bug interno en Bash escrito en C. Lo más común es usar Bash 
    simplemente como \textit{medio} para inyectar la cadena maliciosa a programas en C 
    propensos al desbordamiento.
    
\end{enumerate}
