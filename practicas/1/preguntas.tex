\section{Preguntas de Pentesting: Recopilación de Información}

\subsection{1. Menciona los tipos de protocolos de red.}
Los protocolos de red se pueden clasificar en diferentes capas del modelo OSI, dependiendo de su función:

\textbf{Capa de Enlace de Datos}: Controlan la comunicación dentro de una red local.
\begin{itemize}
    \item Ethernet (IEEE 802.3)
    \item Wi-Fi (IEEE 802.11)
    \item PPP (Point-to-Point Protocol)
\end{itemize}

\textbf{Capa de Red}: Administran la transmisión de datos entre dispositivos en diferentes redes.
\begin{itemize}
    \item IP (Internet Protocol): IPv4, IPv6
    \item ICMP (Internet Control Message Protocol): Diagnóstico y reporte de errores en redes.
    \item ARP (Address Resolution Protocol): Traducción de direcciones IP a direcciones MAC.
\end{itemize}

\textbf{Capa de Transporte}: Garantizan la entrega fiable de datos entre aplicaciones.
\begin{itemize}
    \item TCP (Transmission Control Protocol): Conexión confiable y orientada a flujo.
    \item UDP (User Datagram Protocol): Conexión rápida pero sin garantía de entrega.
\end{itemize}

\textbf{Capa de Aplicación}: Facilitan la comunicación entre aplicaciones.
\begin{itemize}
    \item HTTP/HTTPS (Hypertext Transfer Protocol Secure): Transferencia de información en la web.
    \item FTP (File Transfer Protocol): Transferencia de archivos.
    \item DNS (Domain Name System): Resolución de nombres de dominio en direcciones IP.
    \item SMTP/POP3/IMAP: Protocolos de correo electrónico.
    \item SSH (Secure Shell): Acceso remoto seguro.
\end{itemize}

\textbf{Protocolos de Seguridad}: Proporcionan cifrado y autenticación.
\begin{itemize}
    \item TLS/SSL (Transport Layer Security / Secure Sockets Layer)
    \item IPSec (Internet Protocol Security)
\end{itemize}

\begin{quote}
    "Los protocolos de red definen el conjunto de reglas que permiten la comunicación entre dispositivos interconectados y garantizan la interoperabilidad de los sistemas informáticos" (Kurose \& Ross, 2020).
\end{quote}

\subsection{2. ¿Cómo funcionan los protocolos de red y para qué sirven?}
Los protocolos de red operan en distintos niveles del modelo OSI (Open Systems Interconnection) y del modelo TCP/IP para permitir la transmisión de datos. Su funcionamiento se basa en la encapsulación de información en paquetes que son transportados de un punto a otro dentro de una red.

\textbf{Funcionamiento}: Cada protocolo tiene reglas específicas sobre cómo se deben enviar, recibir y procesar los datos. Por ejemplo, TCP divide la información en paquetes, los transmite y garantiza su correcta entrega y orden.

\textbf{Utilidad}:
\begin{itemize}
    \item Permiten la comunicación entre dispositivos.
    \item Garantizan que los datos lleguen completos y sin alteraciones.
    \item Optimización del uso de recursos de red.
    \item Seguridad y autenticación de datos.
\end{itemize}

\begin{quote}
    "El diseño modular de los protocolos de red permite que la comunicación se mantenga eficiente y escalable, permitiendo su aplicación en distintos escenarios tecnológicos" (Kurose \& Ross, 2020).
\end{quote}

\subsection{3. ¿Qué es un sniffer?}
Un \textit{sniffer} es una herramienta que permite capturar y analizar el tráfico de red en tiempo real. Su función principal es interceptar paquetes de datos que circulan en una red, lo que permite a los administradores de seguridad evaluar la comunicación y detectar vulnerabilidades.

\textbf{Usos de un sniffer}:
\begin{itemize}
    \item Diagnóstico y resolución de problemas de red.
    \item Monitoreo de tráfico para detectar intentos de ataque.
    \item Análisis de vulnerabilidades en entornos de red.
    \item Recuperación de información en auditorías de seguridad.
\end{itemize}

\textbf{Ejemplos de sniffers}:
\begin{itemize}
    \item Wireshark: Herramienta gráfica para análisis profundo del tráfico de red.
    \item tcpdump: Utilidad de línea de comandos para capturar paquetes.
    \item Ettercap: Especializado en ataques MITM (Man-in-the-Middle).
    \item Kismet: Sniffer enfocado en redes inalámbricas.
\end{itemize}

\begin{quote}
    "Los analizadores de paquetes (sniffers) son esenciales para la supervisión del tráfico de red y la identificación de posibles riesgos de seguridad" (Kurose \& Ross, 2020).
\end{quote}

\subsection{4. OSINT, ¿qué es y para qué sirve?}
\textit{OSINT} (Open Source Intelligence) se refiere a la recopilación de información a partir de fuentes abiertas y accesibles al público para obtener inteligencia útil en diversas áreas, incluida la ciberseguridad.

\textbf{Utilidad}:
\begin{itemize}
    \item Recolección de información sobre objetivos en pentesting.
    \item Identificación de activos expuestos en internet.
    \item Seguimiento de actividad sospechosa en redes sociales.
    \item Investigación en ciberseguridad para la detección de amenazas.
\end{itemize}

\begin{quote}
    "La inteligencia de fuentes abiertas (OSINT) permite recopilar datos valiosos a partir de información pública para su posterior análisis y evaluación" (Mitnick \& Simon, 2002).
\end{quote}

\subsection{5. Investiga los 5 OSINT más usados.}
Los cinco \textit{OSINT} más utilizados en ciberseguridad y pruebas de penetración son:

\begin{itemize}
    \item \textbf{Maltego}: Analiza relaciones entre personas, empresas, correos y direcciones IP.
    \item \textbf{Shodan}: Motor de búsqueda para dispositivos conectados a Internet.
    \item \textbf{theHarvester}: Recolecta información sobre correos electrónicos y subdominios.
    \item \textbf{Google Dorking}: Búsqueda avanzada en Google para encontrar información sensible.
    \item \textbf{Recon-ng}: Framework modular para la recolección de información OSINT.
\end{itemize}

\begin{quote}
    "El uso de herramientas OSINT es crucial en la fase de recopilación de información dentro de un test de penetración, ya que permite identificar puntos débiles antes de realizar ataques simulados" (Campbell \& Beach, 2016).
\end{quote}
