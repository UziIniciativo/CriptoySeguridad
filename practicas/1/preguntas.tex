\section{Preguntas de Pentesting: Recopilación de Información}

\subsection{Menciona los tipos de protocolos de red. ¿Cómo funcionan? ¿Para qué sirven?}
Los protocolos de red se pueden clasificar en diferentes capas del modelo OSI, dependiendo de su función:

\textbf{Capa de Enlace de Datos}: Controlan la comunicación dentro de una red local.
\begin{itemize}
    \item Ethernet (IEEE 802.3)
    \item Wi-Fi (IEEE 802.11)
    \item PPP (Point-to-Point Protocol)
\end{itemize}

\textbf{Capa de Red}: Administran la transmisión de datos entre dispositivos en diferentes redes.
\begin{itemize}
    \item IP (Internet Protocol): IPv4, IPv6
    \item ICMP (Internet Control Message Protocol): Diagnóstico y reporte de errores en redes.
    \item ARP (Address Resolution Protocol): Traducción de direcciones IP a direcciones MAC.
\end{itemize}

\textbf{Capa de Transporte}: Garantizan la entrega fiable de datos entre aplicaciones.
\begin{itemize}
    \item TCP (Transmission Control Protocol): Conexión confiable y orientada a flujo.
    \item UDP (User Datagram Protocol): Conexión rápida pero sin garantía de entrega.
\end{itemize}

\textbf{Capa de Aplicación}: Facilitan la comunicación entre aplicaciones.
\begin{itemize}
    \item HTTP/HTTPS (Hypertext Transfer Protocol Secure): Transferencia de información en la web.
    \item FTP (File Transfer Protocol): Transferencia de archivos.
    \item DNS (Domain Name System): Resolución de nombres de dominio en direcciones IP.
    \item SMTP/POP3/IMAP: Protocolos de correo electrónico.
    \item SSH (Secure Shell): Acceso remoto seguro.
\end{itemize}

\textbf{Protocolos de Seguridad}: Proporcionan cifrado y autenticación.
\begin{itemize}
    \item TLS/SSL (Transport Layer Security / Secure Sockets Layer)
    \item IPSec (Internet Protocol Security)
\end{itemize}

\subsection{¿Qué es un sniffer?}
Un \textit{sniffer} es una herramienta que permite capturar y analizar el tráfico de red en tiempo real. Su función principal es interceptar paquetes de datos que circulan en una red, lo que permite a los administradores de seguridad evaluar la comunicación y detectar vulnerabilidades.

\textbf{Usos de un sniffer}:
\begin{itemize}
    \item Diagnóstico y resolución de problemas de red.
    \item Monitoreo de tráfico para detectar intentos de ataque.
    \item Análisis de vulnerabilidades en entornos de red.
    \item Recuperación de información en auditorías de seguridad.
\end{itemize}

\textbf{Ejemplos de sniffers}:
\begin{itemize}
    \item Wireshark: Herramienta gráfica para análisis profundo del tráfico de red.
    \item tcpdump: Utilidad de línea de comandos para capturar paquetes.
    \item Ettercap: Especializado en ataques MITM (Man-in-the-Middle).
    \item Kismet: Sniffer enfocado en redes inalámbricas.
\end{itemize}

\subsection{OSINT, ¿qué es y para qué sirve?}
\textit{OSINT} (Open Source Intelligence) se refiere a la recopilación de información a partir de fuentes abiertas y accesibles al público para obtener inteligencia útil en diversas áreas, incluida la ciberseguridad.

\textbf{Utilidad}:
\begin{itemize}
    \item Recolección de información sobre objetivos en pentesting.
    \item Identificación de activos expuestos en internet.
    \item Seguimiento de actividad sospechosa en redes sociales.
    \item Investigación en ciberseguridad para la detección de amenazas.
\end{itemize}

\subsection{Investiga los 5 OSINT más usados.}
Los cinco \textit{OSINT} más utilizados en ciberseguridad y pruebas de penetración son:

\begin{itemize}
    \item \textbf{Maltego}: Analiza relaciones entre personas, empresas, correos y direcciones IP.
    \item \textbf{Shodan}: Motor de búsqueda para dispositivos conectados a Internet.
    \item \textbf{theHarvester}: Recolecta información sobre correos electrónicos y subdominios.
    \item \textbf{Google Dorking}: Búsqueda avanzada en Google para encontrar información sensible.
    \item \textbf{Recon-ng}: Framework modular para la recolección de información OSINT.
\end{itemize}

\subsection{Investiga 5 softwares no mencionados en la práctica que sirvan para el análisis de comunicaciones.}
Existen diversas herramientas adicionales para el análisis de comunicaciones en redes. Algunas de las más utilizadas en ciberseguridad son:
\begin{itemize}
    \item \textbf{Zeek (Bro)}: Framework de monitoreo de red en tiempo real, útil para análisis de tráfico avanzado.
    \item \textbf{Suricata}: Sistema de detección y prevención de intrusos (IDS/IPS) con capacidad de análisis en tiempo real.
    \item \textbf{TShark}: Versión en línea de comandos de Wireshark, utilizada para captura y análisis de paquetes.
    \item \textbf{NetworkMiner}: Herramienta de análisis forense de red, útil para extraer archivos e información de tráfico capturado.
    \item \textbf{P0f}: Detector pasivo de sistemas operativos y monitoreo de tráfico sin enviar paquetes.
\end{itemize}

\subsection{¿Qué es la ingeniería social?}
La \textbf{ingeniería social} es una técnica de manipulación psicológica utilizada para obtener información confidencial a través del engaño y la persuasión. Se basa en explotar la confianza y el comportamiento humano en lugar de vulnerabilidades técnicas.

Ejemplos de ingeniería social:
\begin{itemize}
    \item \textbf{Phishing}: Suplantación de identidad para obtener credenciales.
    \item \textbf{Pretexting}: Creación de escenarios falsos para obtener información.
    \item \textbf{Baiting}: Uso de dispositivos infectados para comprometer sistemas.
\end{itemize}

\subsection{¿Por qué el eslabón más débil de seguridad son las personas?}
En ciberseguridad, el factor humano es considerado el \textbf{eslabón más débil} porque las personas pueden ser engañadas con mayor facilidad que los sistemas automatizados. Las razones incluyen:
\begin{itemize}
    \item Falta de conocimiento sobre ciberseguridad.
    \item Uso de contraseñas débiles o repetidas.
    \item Caída en engaños de phishing o ingeniería social.
    \item Configuración incorrecta de dispositivos y aplicaciones.
\end{itemize}

\subsection{¿Qué acciones haces para protegerte de ciberataques?}
Para mitigar riesgos de ciberseguridad, se pueden implementar diversas buenas prácticas, como:
\begin{itemize}
    \item Uso de contraseñas seguras y autenticación multifactor (MFA).
    \item Evitar enlaces y archivos sospechosos en correos electrónicos.
    \item Mantener software y sistemas operativos actualizados.
    \item Uso de VPNs y conexiones seguras para acceder a redes sensibles.
    \item Realizar copias de seguridad periódicas.
\end{itemize}

\subsection{¿Crees que tus métodos preventivos son suficientes?}
La prevención en ciberseguridad es un proceso continuo que debe adaptarse a nuevas amenazas. A pesar de implementar medidas de protección, siempre existe la posibilidad de que surjan nuevas vulnerabilidades. Algunas recomendaciones adicionales incluyen:
\begin{itemize}
    \item Realizar auditorías de seguridad periódicas.
    \item Mantenerse informado sobre las últimas amenazas cibernéticas.
    \item Implementar sistemas de detección y respuesta ante incidentes.
\end{itemize}
