% Archivo: fuentes.tex
\section{Identificación de Fuentes de Información}

Para recopilar información en una prueba de penetración, es clave usar fuentes confiables. Aquí hay algunas opciones que pueden aportar datos útiles:

\subsection{Fuentes Públicas y OSINT}
Estas fuentes son accesibles para cualquiera y pueden proporcionar mucha información sin necesidad de permisos especiales:
\begin{itemize}
    \item \textbf{Motores de búsqueda (Google, Bing, DuckDuckGo)}: Si se usan bien, pueden revelar información interesante sobre un sitio web o una empresa.
    \item \textbf{Bases de datos WHOIS}: Permiten ver quién registró un dominio y otros datos relevantes.
    \item \textbf{Shodan y Censys}: Son motores de búsqueda para encontrar dispositivos conectados a Internet.
    \item \textbf{Herramientas OSINT (TheHarvester, Recon-ng)}: Ayudan a recolectar información de correos, subdominios y redes sociales.
\end{itemize}

\subsection{Fuentes Privadas y de Acceso Restringido}
Algunas fuentes requieren credenciales o permisos para acceder:
\begin{itemize}
    \item \textbf{Registros internos}: Logs de servidores y sistemas de una empresa pueden revelar actividad sospechosa.
    \item \textbf{Monitoreo de tráfico de red}: Herramientas como Wireshark permiten analizar paquetes de datos en tiempo real.
    \item \textbf{Sistemas SIEM}: Consolidan eventos de seguridad para facilitar la detección de incidentes.
\end{itemize}

\subsection{Documentación Técnica y Académica}
Libros, artículos y normativas de seguridad son esenciales para entender vulnerabilidades y mejores prácticas:
\begin{itemize}
    \item \textbf{Libros de ciberseguridad}: Como \textit{Computer Networking: A Top-Down Approach} (Kurose \& Ross, 2020).
    \item \textbf{Estándares de seguridad}: NIST, ISO 27001, OWASP Top 10.
    \item \textbf{Blogs y foros}: SANS, KrebsOnSecurity, Exploit-DB.
\end{itemize}

\subsection{Redes Sociales y Foros de Seguridad}
Las comunidades en línea pueden ser útiles para conocer nuevas amenazas y tendencias:
\begin{itemize}
    \item \textbf{GitHub y GitLab}: Repositorios donde se pueden encontrar herramientas de seguridad y exploits.
    \item \textbf{Reddit y Twitter}: Se discuten vulnerabilidades recientes y técnicas de pentesting.
    \item \textbf{Foros especializados (Hack The Box, Offensive Security)}: Espacios donde expertos comparten experiencias y retos.
\end{itemize}   