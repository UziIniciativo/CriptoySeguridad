\section{Introducción}

La seguridad informática es un aspecto crítico en la actualidad, ya que la información es uno de los activos más valiosos de cualquier organización
Las pruebas de penetración, o \textit{pentesting}, son evaluaciones de seguridad diseñadas para identificar vulnerabilidades en sistemas y redes antes de que puedan ser explotadas por actores malintencionados (Mitnick & Simon, 2002).

En esta práctica, se enfoca la fase de \textbf{recopilación de información}, una etapa esencial del \textit{pentesting} en la que se recolectan datos clave sobre el objetivo para evaluar posibles puntos débiles. 
Para ello, se emplearán diversas herramientas y metodologías que permiten obtener información de manera estructurada y efectiva (Campbell & Beach, 2016).

Este documento presenta los requisitos, las fuentes de información utilizadas, el proceso de adquisición de datos, el procesamiento de la información obtenida, el análisis de resultados y una serie de preguntas clave relacionadas con la seguridad informática y el \textit{pentesting}. 
Finalmente, se ofrecerá una conclusión sobre los hallazgos obtenidos y su relevancia en el ámbito de la ciberseguridad.

\bibliographystyle{apa}
\bibliography{referencias}
