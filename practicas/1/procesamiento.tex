\section{Procesamiento}
Retomando nuestros apartados durante la adquisición, filtraremos nuestros resultados de la misma manera.

\subsection{unam.mx}

\begin{enumerate}
    \item Direcciones y dominios
    \begin{enumerate}
        \item \textbf{Dominio:} unam.mx
        \item \textbf{IPv4: 132.248.166.19}
    \end{enumerate}
    \item Fechas importantes
    \begin{enumerate}
        \item Creación: 1989-03-31
        \item Actualización: 2024-03-27
        \item Expiración: 25-03-30
    \end{enumerate}
    \item Datos de la organización.

    \begin{enumerate}
        \item \textbf{Nombre:} Latin American and Caribbean IP address Regional Registry y Universidad Nacional Autónoma de México
        \item \textbf{Países asociados}: México y Uruguay
        \item \textbf{Direcciones Asociadas:} Rambla República de México 6125. Y tambień Av. Universidad, 3000, Copilco 04510, Coyoacán CDMX
        \item \textbf{Personal:} LACNIC, Dr Hector Benitez Perez.
    \end{enumerate}

    \item Conectividad
    \begin{itemize}
        \item Los 4 paquetes fueron recibidos sin fallas con una latencia de 35.309ms
    \end{itemize}

    \item Información de Red y DNS
    \begin{itemize}
        \item La ip cuenta con un segmento de red de 132.247.0.0 - 132.248.255.255
        \item Su dirección IPv6 es 2001::1218:3000:160::19
        \item DNS reverso como 19.166.248.132
        \item Traceroute completado en 3 saltos.
        \item Una variedad de subdomions, entre ellos del tipo blog, bq, dc, email, eventos, fa, ib, im, mail, mobile, ns1, ns2, ns3, pi, ri, servidor, tienda, tv, vpn, ws, www, www1. Juntos a los registros DNS: SOA, NS, MX, A, AAAA, TXT.

        \item Información de puertos, servicios y sistema operativo
        
        \begin{enumerate}
            \item Puertos y servicios:
            \begin{enumerate}
                \item 21/tcp open ftp?
                \item 80/ tcp open http Apache httpd 2.4.51 ((Unix))
                \item 443/tcp open ssl/http nginx 1.27.2
                \item 554/tcp rtsp?
                \item 1723/tcp open pptp?  
            \end{enumerate}

            \item Tipos de dispositivo: Propósito general.
            \item SO: Linux
        \end{enumerate}
    \end{itemize}
\end{enumerate}

\subsection{ipn.mx}

\begin{enumerate}
    \item Direcciones y dominios
    \begin{enumerate}
        \item \textbf{Dominio:} ipn.mx
        \item \textbf{IPv4}: 20.64.80.120
    \end{enumerate}
    \item Fechas importantes
    \begin{enumerate}
        \item Creación:1995-04-30
        \item Actualización:2024-04-26
        \item Expiración:2026-06-20
    \end{enumerate}
    \item Datos de la organización.

    \begin{enumerate}
        \item \textbf{Nombre:} Microsoft Corporation
        \item \textbf{Países asociados}: Estados Unidos 
        \item \textbf{Direcciones Asociadas:} Redmond, Washington, One Microsoft Way 98052
        \item \textbf{Personal:} Dawn Bernard, Avery Kim, Prachi Singh
    \end{enumerate}

    \item Conectividad
    \begin{itemize}
        \item Los 4 paquetes fueron recibidos sin fallas con una latencia de 34.922 ms 
    \end{itemize}

    \item Información de Red y DNS
    \begin{itemize}
        \item La ip cuenta con un segmento de red de 20.33.0.0 - 20.128.255.255
        \item Sin información acerca de IPV6
        \item DNS reverso como 120.80.64.20
        \item Traceroute completado en 30 saltos.
        \item Una variedad de subdominios, entre ellos del tipo backup, home, imap, mail, news, ntp, ok, p, pop, servicios, sg, smtp, soporte, virtual, www. Juntos a los registros DNS: SOA, NS, MX, A, AAAA, TXT.

        \item Información de puertos, servicios y sistema operativo
        
        \begin{enumerate}
            \item Puertos y servicios:
            \begin{enumerate}
                \item 21/tcp open tcpwrapped
                \item 80/ tcp open http tcpwrapped
                \item 443/tcp open tcpwrapped1.27.2
                \item 554/tcp tcpwrapped
                \item 1723/tcp tcpwrapped 
            \end{enumerate}

            \item Tipos de dispositivo: Storage-misc, impresora.
            \item SO: Posiblemente Netgear SC101 Storage Central NAS device
        \end{enumerate}
    \end{itemize}
\end{enumerate}

\subsection{pemex.com}

\begin{enumerate}
    \item Direcciones y dominios
    \begin{enumerate}
        \item \textbf{Dominio:} pemex.com
        \item \textbf{IPv4}: 200.23.91.20
    \end{enumerate}
    \item Fechas importantes
    \begin{enumerate}
        \item Creación:1995-06-22
        \item Actualización:2025-01-07
        \item Expiración:2026-06-20
    \end{enumerate}
    \item Datos de la organización.

    \begin{enumerate}
        \item \textbf{Nombre:} Petróleos Mexicanos
        \item \textbf{Países asociados}: México 
        \item \textbf{Direcciones Asociadas:} Av Marina Nacional, 329, Verónica Anzures, 05200, Miguel Hidal, CDMX
        \item \textbf{Personal:} Rocío Pérez Juárez, Rocio Pérez Juárez, Joel Rne Argaez Soliz.
    \end{enumerate}

    \item Conectividad
    \begin{itemize}
        \item Los 4 paquetes fueron recibidos sin fallas con una latencia de 3078 ms 
    \end{itemize}

    \item Información de Red y DNS
    \begin{itemize}
        \item La ip cuenta con un segmento de red de 132.247.0.0 - 132.248.255.255
        \item Sin información acerca de IPV6
        \item DNS reverso como 20.91.23.200
        \item Traceroute completado en 3 saltos.
        \item Una variedad de subdomions, entre ellos del tipo blog, blogs, ca, email, eventos, ir, owa, ri, search,tv, vpn, ws, www, www1, ww2. Juntos a los registros DNS: SOA, NS, MX, A, AAAA, TXT.

        \item Información de puertos, servicios y sistema operativo
        
        \begin{enumerate}
            \item Puertos y servicios:
            \begin{enumerate}
                \item 21/tcp open tcpwrapped
                \item 80/ tcp open http tcpwrapped
                \item 443/tcp open tcpwrapped1.27.2
                \item 554/tcp tcpwrapped
                \item 1723/tcp tcpwrapped 
            \end{enumerate}

            \item Tipos de dispositivo: Storage-misc, impresora.
            \item SO: Posiblemente Netgear SC101 Storage Central NAS device
        \end{enumerate}
    \end{itemize}
\end{enumerate}


\subsection{gob.mx}

No hay dirección IP en un servidor DNS autorativo.