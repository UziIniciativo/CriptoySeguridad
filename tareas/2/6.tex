Determina si $a$ es residuo cuadrático módulo $n$. Muestra tu procedimiento.

\begin{enumerate}
    \item $a = 6007$, $n=1902$
    
    Se factoriza $n = 2 \cdot 3 \cdot 317$, por lo que usamos el símbolo de Jacobi:

    \[
    \left( \frac{6007}{1902} \right) = \left( \frac{6007}{2} \right) \left( \frac{6007}{3} \right) \left( \frac{6007}{317} \right)
    \]

    Calculamos cada uno:
    \begin{itemize}
    \item $6007 \mod 8 = 7 \Rightarrow \left( \frac{6007}{2} \right) = 1$ (pues $7 \equiv 7 \mod 8$)
    \item $6007 \mod 3 = 1 \Rightarrow \left( \frac{6007}{3} \right) = \left( \frac{1}{3} \right) = 1$
    \item $\left( \frac{6007}{317} \right) = 1$ (Fendt, s. f.)
    \end{itemize}


    \[
    \left( \frac{6007}{2} \right) = 1, \quad \left( \frac{6007}{3} \right) = 1, \quad \left( \frac{6007}{317} \right) = 1
    \]

    \[
    \left( \frac{6007}{1902} \right) = 1 \cdot 1 \cdot 1 = 1
    \]

    El símbolo de Jacobi da $1$, lo cual sugiere que podría ser un residuo cuadrático, pero no garantiza que lo sea (pues $n$ no es primo).

        \item $a=83$, $n=593$
        
        Como $n$ es primo, se usa el símbolo de Legendre:

        \[
    \left( \frac{83}{593} \right) = 1
    \]

    Entonces, $83$ es un residuo cuadrático módulo $593$. (Fendt, s. f.)

    \item $a=3677176$, $n=4568731$
    
    Como $n$ es primo:

    \[
    \left( \frac{3677176}{4568731} \right) = 1
    \]

    Por lo tanto, $3677176$ es un residuo cuadrático módulo $4568731$. (Fendt, s. f.)

    \item $a=4568723$, $n=4568731$
    
    Nuevamente $n$ es primo:

    \[
    \left( \frac{4568723}{4568731} \right) = 1
    \]

    Entonces, $4568723$ es un residuo cuadrático módulo $4568731$. (Fendt, s. f.)

    Los cálculos se apoyaron en una herramienta online para obtener los símbolos de Legendre y Jacobi \cite{fendt_legendre}.

\end{enumerate}